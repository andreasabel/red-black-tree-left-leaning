\documentclass{scrartcl}
\usepackage{ucs}
\usepackage[utf8x]{inputenc}
\usepackage{verbatim}
\usepackage{autofe}
\usepackage{mdwlist}
\usepackage{mathabx}

\newcommand\textbeta{\ensuremath{\beta}}
\newcommand\textgamma{\ensuremath{\gamma}}

\DeclareUnicodeCharacter{"225F}{$\stackrel{\tiny ?}{=}$}
\DeclareUnicodeCharacter{"2237}{\ensuremath{::}}
\DeclareUnicodeCharacter{"02E1}{\ensuremath{^l}}
\DeclareUnicodeCharacter{"220E}{\ensuremath{\sqbullet}}

\setlength{\parindent}{0cm}
\setlength{\parskip}{3ex plus 2ex minus 2ex}

\newenvironment{code}{\verbatim}{\endverbatim}

\begin{document}

\title{Deletion in Left-leaning Red-Black trees, a proven functional approach in Agda}
\date{\today}
\author{Julien Oster, Ludwig-Maximilians-Universität München}

\maketitle

\section{The data type for LLRBs}

The data type used for representing Left-leaning Red-Black trees is
directly taken from \cite{eha2009}, with minor modifications. It uses
internal logic as a technique to embed all invariants that make up a
valid Left-leaning Red-Black tree in the data type and its
constructors itself. Thus, every function that creates or transforms
LLRB trees using this data type will only type check if every
invariant is proven as part of its definition.

While we present the tree data type in full detail, we will also look
at the various differences to the original data type as specified in
\cite{eha2009}.

First, let us declare the module in which the data type and
all code is contained.

\begin{code}
module LLRBTree2
  (order : StrictTotalOrder Level.zero Level.zero Level.zero) where
\end{code}
Herein, we see the first distinction to \cite{eha2009}, which aims to
fit the code more into the Standard Library. The original code
parameterized the module with:

\begin{itemize*}
\item a type (the key type),
\item an order and an equivalence relation onto the key type,
\item a function for comparing values,
\item another function for applying the transitivity of the order relation,
\end{itemize*}

This is done to allow any type to fill out the role as a key, as long
as we can also provide the other parameters to it. However, it's very
often the case that a satisfying data type brings its common order
relation with it, together with a function to compare and means to
apply transitivity. For example, on the type of natural numbers this
would be its common strict total order relation $<$, which orders
natural numbers from smaller to greater.

We therefore choose to use a single \verb/StrictTotalOrder/ (as
defined in the Standard Library) as a parameter instead. This
particular record type fits well because it contains
all of the above as fields of itself.

By subsequently opening the record, all relevant fields (such as
\verb/_<_/, \verb/_≟_/, \verb/trans/, \verb/compare/) are available to
us in the namespace of our module, without having to specify the name
\verb/order/:

\nopagebreak
\begin{code}
open module sto = StrictTotalOrder order
A = StrictTotalOrder.Carrier order
\end{code}
The second line allows us to refer to the \verb/StrictTotalOrder/'s
carrier, which is the type used to store the node's keys, simply as
\verb/A/ in our namespace.

Next, we define what bounds are:

\begin{code}
BoundsL = List A
BoundsR = List A
\end{code}

Bounds are just simple lists of values (of our key type, \verb/A/). A
value being within those bounds means that it is smaller
(\verb/BoundsR/, ``right'' bounds) or greater (\verb/BoundsL/,
``left'' bounds) than all elements in the respective list. However,
this simple list type doesn't reflect that yet. To achieve this, we
want to get types for proofs out of this bounds.

These proofs are what keeps our trees consistent with the search tree
invariant: The tree type will be parameterized with each of both
bounds lists. Then, any node constructor which contains a key
(i.e. any constructor except for the leaf constructor) must contain
proofs that its key is within \verb/BoundsL/ and
\verb/BoundsR/. Moreover, every non-leaf constructor which creates a
node also specifies a left and a right subtree as children, but the
bounds list in the type of those children (\verb/BoundsR/ for the left
child, \verb/BoundsL/ for the right child) is the parent's bounds
list, prepended with the key of this node. Conclusively, every subtree
must now not only satisfy the original bounds of its parent node, but
additionally all keys within it must now also be smaller (if it is a
left subtree) or greater (if it is a right subtree) than its parent's
key!

The \emph{types} of these proofs are gained with the following operators:

\begin{code}
infix 5 _isleftof_
_isleftof_ : A → BoundsR → Setℓ
z isleftof []     = ⊤
z isleftof b ∷ β  = z < b × z isleftof β

infix 5 _isrightof_
_isrightof_ : A → BoundsL → Setℓ
z isrightof []     = ⊤
z isrightof b ∷ γ  = b < z × z isrightof γ
\end{code}

Looking just at \verb/BoundsR/, proving that a value is within
particular bounds means supplying a proof that the value is smaller
than the head of its bounds list and, inductively, that it's within the
rest of the bounds list. A value is always within empty bounds (thus
the type is the trivially provable ⊤ in that case).

The same applies to \verb/BoundsL/, except that the list's head and
the given value are now reversed in the relation (thus, the value must
be greater).

A key difference between the bounds definition in \cite{eha2009} and
this one is that there are now two bounds lists instead of just
one. \cite{eha2009} defined the single bounds type to contain either
constraints in any order, by not keeping simple lists of values, but
having different constructors for bound values that are to be ``left''
and ``right'' bounds. Subsequently, the tree type was only
parameterized with the one bounds list and each inner node constructor
only needed to specifies one proof instead of two, the proof that its key
value is within the given bounds.

The reason why we chose to abandon this approach in favor of two
explicit lists of bounds will get clear after we look at the
reasons for handling transformation of them:

\begin{code}
infix 5 _⇒ʳ_
data _⇒ʳ_ : BoundsR → BoundsR → Setℓ where
  ∎      : ∀ {γ} → γ ⇒ʳ γ
  keep_  : ∀ {γ γ' b} → γ ⇒ʳ γ' → b ∷ γ ⇒ʳ b ∷ γ'
  skip_  : ∀ {γ γ' b} → γ ⇒ʳ γ' → b ∷ γ ⇒ʳ γ'
  cover_,_  : ∀ {β β' x y} → x < y → x ∷ y ∷ β ⇒ˡ β'
         → x ∷ β ⇒ˡ β'

infix 5 _⇒ˡ_
data _⇒ˡ_ : BoundsL → BoundsL → Setℓ where
  ∎      : ∀ {β} → β ⇒ˡ β
  keep_  : ∀ {β β' b} → β ⇒ˡ β' → b ∷ β ⇒ˡ b ∷ β'
  skip_  : ∀ {β β' b} → β ⇒ˡ β' → b ∷ β ⇒ˡ β'
  cover_,_  : ∀ {γ γ' x y} → x < y → y ∷ x ∷ γ ⇒ʳ γ'
         → y ∷ γ ⇒ʳ γ'

\end{code}

Transforming bounds happens because when reconfiguring a tree into
another, the bounds of individual nodes may change, while the overall
relation between them stays intact (otherwise, we would break the
search invariant).

Let's look at a valid transformation of a simple tree:

\begin{verbatim}
                    b
     c             a c
   b   d    -->       d
  a     f              e
       e g              f
                         g
\end{verbatim}

It is clear that while the shapes of the trees are vastly different, they
still retain the same order relations between their elements and thus
the search tree invariant has been kept.

However, let's look at the bounds list of node \verb/e/'s tree
type. We hereby assume that the bounds for the root element of the
left tree, \verb/c/, are the represented by the unknown lists β and γ
for left and right bounds, respectively. In the left tree, the bounds
lists look like that:

\begin{verbatim}
left:  c :: d :: β
right: f :: γ
\end{verbatim}

With every level down, the parent's key is prepended to the respective
bounds list (depending on whether the child is left or right). Thus,
according to the resulting lists, \verb/e/ must be smaller than
\verb/f/, but greather than \verb/c/ and \verb/d/. \verb/a/, \verb/b/
and \verb/g/ are not appearing because they are not on the path
between the root and \verb/e/, and thus never handed down.

On the right tree, the bounds lists are:

\begin{verbatim}
left:  b :: c :: d :: β
right: γ
\end{verbatim}

Node \verb/f/, formerly a bound, is now a child of \verb/e/, and thus
fell out of its bounds (the roles have been reversed: \verb/f/ must
now satisfy \verb/e/'s bounds). In fact, along the path from the root
to \verb/e/, \verb/e/ is now the node with the largest key, and the
right bounds list has simply become γ, the right bounds that apply to
the whole tree.

On the other hand, due to the rotation of \verb/b/ into the path as
the new root node, \verb/b/ now appears in the left bounds list.

The rules defined above exist to satisfy the need of showing that the
bounds of a node's type are applicable to in a differently configured
tree. They are a \emph{series of rules} that are stringed together to
form a proof that the original list of bounds satisfies the new one,
which is manifested by the node's type at its place in the new tree
(recall that the parent node essentially dictates the type of its
children).

For ease of explanation, we discard the notion that the new bounds
list manifests elsewhere and merely needs to be proven equivalent, and
regard them as if they were operations to transform the original list
into the new one instead:

\begin{description}
\item[keep] Keep the current bound and move on to the next.
\item[skip] Skip the current bound, i.e. it is not needed anymore
  (possibly because its node has been moved out of the path, or
  completely removed).
\item[cover] Insert a new bound after the current bound, i.e. a new
  node appeared along the path to the root (similarly to above, it may
  have been moved in or is a new node). This is allowed in the case
  that the current bound is a \emph{better} bound than the new one:
  Let $a$ be any left bound for $x$. If $b<a$, then $b<x$ due to
  transitivity, and $b$ is also a left bound for $x$. Analog with
  right bounds. This operation requires us to specify a proof of
  $b<a$. The current bound is still the former, better bound, so that
  it may be involved in another step.
\item[∎] Keep the rest of the bounds, starting from the current one,
  as it is.
\end{description}

Going back to the example above, we can transform the right bounds
with \verb/skip ∎/ and the left bounds with
\verb/cover b<c , keep keep ∎/, or simply \verb/cover b<c , ∎/, and we
know that the proof \verb/b<c/ exists in the former tree, from the fact
that \verb/b/ is a left child of \verb/c/.

\pagebreak

Another example illustrates the decision for representing the bounds
with two lists instead of one. Given the following tree
transformation:

\begin{verbatim}

    b           d
  a   d  -->  b   e
     c e     a c

\end{verbatim}

Despite the transformation, the bounds for node \verb/c/ in both trees are:

\begin{verbatim}
left:  b :: β
right: d :: γ
\end{verbatim}

Whereas the representation used in \cite{eha2009}, which
effectively denotes the path back to the root, would yield to the
differing lists \verb/leftOf d :: rightOf b :: β/ for the left tree
and \verb/rightOf b :: leftOf d :: β/ for the right tree.

This necessitated the need for another rule, the \verb/swap/ rule,
which was used to prove that the position of bounds inside the list
was irrelevant by allowing to swap the position of two consecutive
bounds. However, in more complex transformations that change the shape
of a tree significantly, parent nodes can change their position by
more than one tree level (while still staying in a node's path to the
root), resulting in a lot of applications of \verb/swap/.

Moreover, in the case that only left or only right bounds changed,
superfluous \verb/keep/ applications were necessary to disregard
bounds of the other kind.

With separate bounds, the invariant ordering of the lists renders
\verb/swap/ futile, and we may look at only one kind of bounds.


\bibliographystyle{plain}
\bibliography{llrb.bib}

\end{document}
